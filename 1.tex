\documentclass{article}
\usepackage[a4paper,left=3.5cm,right=2.5cm,top=2.5cm,bottom=2.5cm]{geometry}
\usepackage[MeX]{polski}
\usepackage[cp1250]{inputenc}
\usepackage{polski}
\usepackage{graphicx}
\usepackage[table]{xcolor}
\usepackage{booktabs}
 %%\usepackage[utf8]{inputenc}
\usepackage[pdftex]{hyperref}
\usepackage{makeidx}
\usepackage{sidecap}
\usepackage{wrapfig}
\usepackage{caption}
\usepackage{subcaption}
\usepackage[tableposition=top]{caption}
\usepackage{algorithmic}
\usepackage{enumerate}
\usepackage{multirow}
\usepackage{amsmath} %pakiet matematyczny
\usepackage{amssymb} %pakiet dodatkowych symboli
\begin{document}
\begin{table}
\centering	
		\begin{tabular}{|c|c|c|c|c|c|c|}
		\hline
		\multirow{2}{*}{No. of visual words}
		& \multicolumn{6}{c|}{Dataset} \\ \cline{2-7}
      & 1 & 2 & 3 & 4 & 5 & 6 \\ \hline
		50 & 61.27\% & 88.92\% &77.88\%& 87.89\%
		& 92.04\% & 96.65 \% \\ \hline
		100 & 64.60\% & 89.40\% &80.41\%& 92.05\%
		& 95.81\% & 97.12 \% \\ \hline
		\end{tabular}
	\caption{final classifcation}
	\label{tab:final}
\end{table}
\begin{figure}[h!]
\caption{Ogórek pokrojony}
\centering
  \includegraphics[width=0.5\textwidth]{obrazek1.jpg}
	\end{figure}
	\begin{figure}[h!]
\caption{Pomidor}
\centering
  \includegraphics[width=0.5\textwidth]{obrazek2.jpg}
	\end{figure}	
	\begin{figure}[h!]
	\centering
	\reflectbox{
\includegraphics[width=0.5\textwidth]{obrazek2.jpg}}
	\caption{ten sam pomidor w drugą strone}
	\end{figure}
		\begin{figure}[here]
		\vspace{0pt}
		\begin{center}
		\includegraphics[scale=0.2]{obrazek2.jpg}
		\reflectbox{
		\includegraphics[scale=0.3]{obrazek2.jpg}}
		\includegraphics[scale=0.4]{obrazek2.jpg}
		\reflectbox{
		\includegraphics[scale=0.5]{obrazek2.jpg}}
		\caption{POMIDORY!!}
		\end{center}
		\vspace{0pt}
		\end{figure}
		
		\begin{table}
			\centering
				\begin{tabular}{c|ccc}
				\hline
				\hline
				$Pacjent$ & $Bol brzucha$ & $Temepratura ciala$ & $Operacja$ \\ \hline
				$u1$& $mocny$ & $Wysoka$ & $Tak$  \\
				$u2$& $sredni$& $Wysoka$ & $Tak$ \\
				$u3$& $mocny$ & $srednia$& $Tak$  \\
				$u4$& $mocny$ & $Niska$ & $Tak$  \\
				$u5$& $sredni$& $sredna$ & $Tak$  \\
				$u6$& $mocny$ & $Wysoka$ & $Tak$  \\
				$u7$& $mocny$ & $Wysoka$ & $nie$  \\
				$u8$& $mocny$ & $Wysoka$ & $nie$  \\
				$u9$& $mocny$ & $Wysoka$ & $nie$  \\
				$u1$& $mocny$ & $Wysoka$ & $nie$  \\
				\hline
				\hline
		\end{tabular}
		\end{table}
		
		\begin{table}
			\centering
				\begin{tabular}{c|c|c}
				\hline
				\hline
			$x_{1}$&$x_{2}$&$(x_{1}ANDx_{2})$\\ \hline
					1 & 1 & 1 \\
					1 & 0 & 0 \\
					0 & 1 & 0 \\
					0 & 0 & 0 \\
					\hline
					\hline
		\end{tabular}
		\end{table}
		
		\begin{table}
\centering	
	\rowcolors{1}{yellow}{orange}
		\begin{tabular}{|c|c|}
	\hline
		$7c0$ & $hexadecimal$ \\
		$3700$ & $octal$ \\
		$11111000000$ & $binary$ \\ \cline{2-2}
		\hline
		\hline
		$1984$ & $decimal$\\
		\hline
		\end{tabular}
		\end{table}
		
		\begin{table}
			\centering
				\begin{tabular}{|c|c|c|c|}
				\hline
			  \multicolumn{4}{|c|}{Primcs} \\ \hline
				2 & 3 & 5 & 7 \\ \hline
			  \multirow{2}{*}{Powers} 
				
				
				
					
				\end{tabular}
		\end{table}
		
	\end{document}
